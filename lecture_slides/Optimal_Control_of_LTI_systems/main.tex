\documentclass{beamer}
\usetheme{Madrid}


 \pdfmapfile{+sansmathaccent.map}
 
\renewcommand{\familydefault}{\rmdefault}


\usepackage{amsmath}
\usepackage{mathtools}

\DeclareMathOperator*{\argmin}{arg\,min}

\title{Optimal Control of LTI systems}
\author{by Sergei Savin}
\centering
\date{Spring 2020}


\begin{document}
\maketitle


\begin{frame}{Content}
\begin{itemize}
\item Hamilton-Jacobi-Bellman equation
\item LQR
\item Algebraic Riccati equation
\end{itemize}
\end{frame}

\begin{frame}{Hamilton-Jacobi-Bellman equation}
\framesubtitle{Cost and optimal cost}
\begin{flushleft}

Let as define dynamics:

\[
\dot {\mathbf x} = \mathbf f (\mathbf x, \mathbf u)
\]
%
with initial conditions $\mathbf x(0)$. Let $J$ be an additive cost function:

\[
J (\mathbf x(0), \mathbf u(t)) = \int_0^\infty g(\mathbf x, \mathbf u) dt
\]
%
where $g(\mathbf  x, \mathbf  u)$ is instantaneous cost.

\bigskip
\bigskip

Let $J^*$ be the optimal (lowest possible) cost. In other words:

\[
J^*(\mathbf x(0)) = \underset{\mathbf u(t)}{\inf{}} J(\mathbf x(0), \mathbf u(t))
\]


\end{flushleft}
\end{frame}

\begin{frame}{Hamilton-Jacobi-Bellman equation}
\framesubtitle{Differentiating optimal cost}
\begin{flushleft}
Since $J^*(\mathbf x(0))$ does not depend on $t$, its full derivative is zero:

\[
\frac{d J^*(\mathbf x(0))}{dt} = 0
\]

At the same time, we can expand the full derivative as follows:

\[
\frac{d J^*}{dt } = 
\frac{\partial J^*}{\partial \mathbf x} \dot {\mathbf x} +
\frac{\partial J^*}{\partial t} = 0
\]

Observe that $\frac{\partial J^*}{\partial t} = g(\mathbf x, \mathbf u)$, and $\dot {\mathbf x} = \mathbf f (\mathbf x, \mathbf u)$. Therefore:

\[
\frac{\partial J^*}{\partial \mathbf x} \mathbf f (\mathbf x, \mathbf u) +
g(\mathbf x, \mathbf u) = 0
\]

\end{flushleft}
\end{frame}




\begin{frame}{Hamilton-Jacobi-Bellman equation}
\framesubtitle{HJB}
\begin{flushleft}

With this, we can formulate Hamilton-Jacobi-Bellman equation (HJB):

\[
\underset{\mathbf u}{\min} \ [ g(\mathbf x, \mathbf u) + 
\frac{\partial J^*}{\partial \mathbf x} \mathbf f (\mathbf x, \mathbf u) ] = 0
\]

And therefore:

\[
u^* = \underset{\mathbf u}{\argmin} \ [ g(\mathbf x, \mathbf u) + 
\frac{\partial J^*}{\partial \mathbf x} \mathbf f (\mathbf x, \mathbf u) ]
\]

\end{flushleft}
\end{frame}

\begin{frame}{Algebraic Riccati}
\framesubtitle{HJB for LTI}
\begin{flushleft}

For LTI, dynamics is:
\[
\dot {\mathbf x} = \mathbf A  \mathbf x + \mathbf B \mathbf u
\]

We can choose quadratic cost:
\[
g(\mathbf  x, \mathbf  u) = 
\mathbf  x^\top \mathbf Q \mathbf x +
\mathbf  u^\top \mathbf R \mathbf u 
\]

Then HJB becomes:
\[
\underset{\mathbf u}{\min} \ [ 
\mathbf  x^\top \mathbf Q \mathbf x +
\mathbf  u^\top \mathbf R \mathbf u + 
\frac{\partial J^*}{\partial \mathbf x} 
(\mathbf A  \mathbf x + \mathbf B \mathbf u)] = 0
\]
%
where $\mathbf Q = \mathbf Q^\top \geq 0 $ and $\mathbf R = \mathbf R^\top > 0$.

\end{flushleft}
\end{frame}


\begin{frame}{Algebraic Riccati}
\framesubtitle{HJB for LTI, part 2}
\begin{flushleft}

There is a theorem that says that for LTI with quadratic cost, $J^*$ has the form:

\[
J^* = \mathbf  x^\top \mathbf S \mathbf x
\]
%
where $\mathbf S = \mathbf S^\top \geq 0$.

\bigskip

Then HJB becomes:
\[
\underset{\mathbf u}{\min} \ [ 
\mathbf  x^\top \mathbf Q \mathbf x +
\mathbf  u^\top \mathbf R \mathbf u + 
2\mathbf x^\top \mathbf S
(\mathbf A  \mathbf x + \mathbf B \mathbf u)] = 0
\]

\end{flushleft}
\end{frame}


\begin{frame}{Algebraic Riccati}
\framesubtitle{LQR}
\begin{flushleft}


Finding partial derivative of the HJB with respect to $\mathbf u$ and setting it to zero (as it is an extreme point) we get:
\[
2 \mathbf  u^\top \mathbf R + 
2 \mathbf x^\top \mathbf S \mathbf B = 0
\]

This expression can be transposed and $\mathbf  u$ separated:
\[
\mathbf  u = 
-\mathbf R^{-1} \mathbf B^\top \mathbf S \mathbf x
\]

This is the desired control law. We can see that it is \emph{proportional}. We can re-write it as:
\[
\mathbf  u = -\mathbf K \mathbf x
\]
where $\mathbf K = \mathbf R^{-1} \mathbf B^\top \mathbf S$ is the controller gain. This control law is called Linear Quadratic Regulator (LQR).

\end{flushleft}
\end{frame}


\begin{frame}{Algebraic Riccati}
\framesubtitle{Algebraic Riccati}
\begin{flushleft}

Substituting found control law into the HJB, we find:
\[
\mathbf  x^\top \mathbf Q \mathbf x +
\mathbf x^\top \mathbf S \mathbf B \mathbf R^{-1} \mathbf R \mathbf  R^{-1} \mathbf B^\top \mathbf S \mathbf x + 
2\mathbf x^\top \mathbf S
(\mathbf A  \mathbf x - \mathbf B \mathbf R^{-1} \mathbf B^\top \mathbf S \mathbf x) = 0
\]

After grouping the terms, we get:
%
\[
\mathbf  x^\top (\mathbf Q - \mathbf S \mathbf B \mathbf  R^{-1} \mathbf B^\top \mathbf S + 
2 \mathbf S \mathbf A) \mathbf x = 0
\]
%
which would hold for all $\mathbf x$ iff:
%
\[
\mathbf Q - \mathbf S \mathbf B \mathbf  R^{-1} \mathbf B^\top \mathbf S + 2 \mathbf S \mathbf A = 0
\]

This is the Algebraic Riccati equation.

\end{flushleft}
\end{frame}

\begin{frame}{Algebraic Riccati}
\framesubtitle{Numerical methods}
\begin{flushleft}

There are a number of ways to solve LQR:
\begin{itemize}
    \item In MATLAB there is a function \texttt{[K,S] = lqr(A, B, Q, R)}
    \item In Python, there is \texttt{S = scipy.linalg.solve\_continuous\_are(a, b, q, r)}
    \item In Drake (by MIT and Toyota Research) there is a function \texttt{(K,S) = LinearQuadraticRegulator(A,B,Q,R)}
\end{itemize}

\end{flushleft}
\end{frame}



\begin{frame}
\centerline{Lecture slides are available via Moodle.}
\bigskip
\centerline{Check Moodle for additional links, videos, textbook suggestions.}
\end{frame}

\end{document}
