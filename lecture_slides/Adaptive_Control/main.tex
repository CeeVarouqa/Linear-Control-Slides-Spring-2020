\documentclass{beamer}

 \pdfmapfile{+sansmathaccent.map}


\mode<presentation>
{
  \usetheme{Madrid}      % or try Darmstadt, Madrid, Warsaw, ...
%   \usecolortheme{beaver} % or try albatross, beaver, crane, ...
  \usefonttheme{serif}  % or try serif, structurebold, ...
  \setbeamertemplate{navigation symbols}{}
  \setbeamertemplate{caption}[numbered]
} 


\renewcommand{\familydefault}{\rmdefault}


\usepackage{amsmath}
\usepackage{mathtools}

\DeclareMathOperator*{\argmin}{arg\,min}

\title{Adaptive Control}
\author{by Sergei Savin}
\centering
\date{Spring 2020}


\begin{document}
\maketitle


\begin{frame}{Content}
\begin{itemize}
\item Feed-forward Adaptive Control
\begin{itemize}
\item Parameter estimation
\item FF control of mechanical systems
\end{itemize}
\item Feedback Adaptive Control
\begin{itemize}
\item LQR
\end{itemize}
\item Real-time Parameter Estimation
\begin{itemize}
\item Linear feedback form
\item Parameter estimation error dynamics
\item Parameter estimation error dynamics stability
\end{itemize}
\end{itemize}


\end{frame}

\begin{frame}{Feed-forward Adaptive Control}
\framesubtitle{Parameter estimation}
\begin{flushleft}

Last time we learned that we can represent a system in the form:
%
\[
\mathbf y = \mathbf M (\mathbf x, \dot {\mathbf x}) \theta
\]
%
and then collect data and solve it for $\theta$ via Linear Least Squares or, equivalently, we pseudoinverse. 

\bigskip

Let us demonstrate how it can be used to generate feedback control.

\end{flushleft}
\end{frame}

\begin{frame}{Feed-forward Adaptive Control}
\framesubtitle{FF control of mechanical systems}
\begin{flushleft}

One example of a system for which we can do it is the following:
\[
\mathbf H \ddot {\mathbf q} + \mathbf c(\dot {\mathbf q}, \mathbf q) = \mathbf u
\; \; \; \Leftrightarrow \; \; \; 
\begin{cases}
\mathbf y = \mathbf Y (\ddot {\mathbf q}, \dot {\mathbf q}, \mathbf q) \theta \\
\mathbf u \equiv \mathbf y
\end{cases}
\]
%
(see previous lecture for notation and derivation)

\bigskip

Assume we have desired trajectory $\mathbf q^* = \mathbf q^*(t)$ and we found precise values of $\theta$. Then we can define control as follows:
%
\[
\mathbf u = \mathbf K_p (\mathbf q^* - \mathbf q) +
            \mathbf K_d (\dot{\mathbf q}^* - \dot{\mathbf q}) +
            \mathbf Y (\ddot {\mathbf q}^*, \dot {\mathbf q}, \mathbf q) \theta
\]

where 
\[
\mathbf Y (\ddot {\mathbf q}^*, \dot {\mathbf q}, \mathbf q) \theta = 
\mathbf H \ddot {\mathbf q}^* + \mathbf c(\dot {\mathbf q}, \mathbf q)
\]

\end{flushleft}
\end{frame}


\begin{frame}{Feed-forward Adaptive Control}
\framesubtitle{FF control of mechanical systems}
\begin{flushleft}

Let's prove that the proposed control is stable. We know that:
%
\[
\begin{cases}
\mathbf H \ddot {\mathbf q} + \mathbf c(\dot {\mathbf q}, \mathbf q) = \mathbf u \\
\mathbf u = \mathbf K_p (\mathbf q^*- \mathbf q) +
            \mathbf K_d (\dot{\mathbf q}^*- \dot{\mathbf q}) +
            \mathbf H \ddot {\mathbf q}^* + \mathbf c(\dot {\mathbf q}, \mathbf q)
\end{cases}
\]
%
then, introducing $\mathbf e = \mathbf q^* - \mathbf q$, and choosing $\mathbf K_p = \mathbf H \mathbf D_p$ and $\mathbf K_d = \mathbf H \mathbf D_d$, where $\mathbf D_p > 0$ and $\mathbf D_d > 0$ are positive-definite diagonal matrices:
\[
\mathbf H \ddot {\mathbf q}^* - \mathbf H \ddot {\mathbf q} +
\mathbf c(\dot {\mathbf q}, \mathbf q) - 
\mathbf c(\dot {\mathbf q}, \mathbf q) + 
\mathbf K_p (\mathbf q^*- \mathbf q) +
\mathbf K_d (\dot{\mathbf q}^*- \dot{\mathbf q}) = \mathbf 0
\]
\[
\mathbf H \ddot {\mathbf e} + \mathbf K_d \dot{\mathbf e} + \mathbf K_p \mathbf e = \mathbf 0
\]
\[
\ddot {\mathbf e} + \mathbf D_d \dot{\mathbf e} + \mathbf D_p \mathbf e = \mathbf 0
\]

And thus we got a system of independent second order linear differential equations with strictly positive coefficients, which are stable:
\[
\ddot {e}_i + d_{d, i} \dot{e}_i + d_{p, i} e_i = 0
\]



\end{flushleft}
\end{frame}


\begin{frame}{Feedback Adaptive Control}
\framesubtitle{LQR}
\begin{flushleft}

Assume we had a linear system, whose model is a function of the estimated parameters $\theta$:

\[
\mathbf A = \mathbf A(\theta),  \; \; \mathbf B = \mathbf B(\theta)
\]

Then if we know a good estimate of the parameters $\Tilde{\theta}$, we can use matrices $\mathbf A$ and $\mathbf B$ to solve Riccati eq.

\[
\begin{cases}
\mathbf Q - \mathbf S \mathbf B(\Tilde{\theta}) \mathbf  R^{-1} \mathbf B^\top(\Tilde{\theta}) \mathbf S + 2 \mathbf S \mathbf A(\Tilde{\theta}) = 0 \\
\mathbf  u = -\mathbf R^{-1} \mathbf B^\top(\Tilde{\theta}) \mathbf S \mathbf x
\end{cases}
\]

However, this guarantees neither optimality, nor stability.

\end{flushleft}
\end{frame}


\begin{frame}{Real-time Parameter Estimation}
\framesubtitle{Linear feedback form}
\begin{flushleft}

In order to use control methods discussed previously, we need to be able to estimate parameters on the fly. This is especially true if the parameters are changing in time.

We can estimate parameters in real time same as we did with state estimation, using a linear feedback law:

\[
\frac{d}{dt}\Tilde{\theta} = \Gamma \mathbf M^\top \mathbf e_{\theta}
\]
%
where $\mathbf M$ is a regressor matrix. 

\end{flushleft}
\end{frame}



\begin{frame}{Real-time Parameter Estimation}
\framesubtitle{Parameter estimation error dynamics}
\begin{flushleft}

Since $\mathbf e_{\theta} = \mathbf y - \mathbf M \Tilde{\theta}$, we have:
%
\[
\frac{d}{dt}\Tilde{\theta} = \Gamma \mathbf M^\top (\mathbf y - \mathbf M \Tilde{\theta})
\]

By definition $\varepsilon_{\theta} = \theta - \Tilde{\theta}$ and $\frac{d}{dt}\varepsilon_{\theta} = \frac{d}{dt}\theta - \frac{d}{dt}\Tilde{\theta}$, and if we assume that $\frac{d}{dt}\theta = 0$, we get:
%
\[
\frac{d}{dt}\varepsilon_{\theta} = \Gamma \mathbf M^\top (\mathbf M (\theta - \varepsilon_{\theta}) - \mathbf y)
= \Gamma \mathbf M^\top (\mathbf M \theta - \mathbf M \varepsilon_{\theta} - \mathbf M \theta)
\]
%
where we used the fact that by definition $\mathbf M \theta = \mathbf y$. Finally, we get the parameter estimation error dynamics equation:
%
\[
\frac{d}{dt}\varepsilon_{\theta}
= - \Gamma \mathbf M^\top \mathbf M \varepsilon_{\theta}
\]

\end{flushleft}
\end{frame}



\begin{frame}{Real-time Parameter Estimation}
\framesubtitle{Parameter estimation error dynamics stability}
\begin{flushleft}

In order to achieve correct estimates of parameters, parameter estimation error dynamics needs to be stable:
\[
- \Gamma \mathbf M^\top \mathbf M  < 0
\]

This means that we need to find such $\Gamma$ that $\Gamma \mathbf M^\top \mathbf M > 0$. This is a non-trivial task. If $\text{null} (\mathbf M^\top \mathbf M)  \neq 0$, then it is generally impossible.

\bigskip

If $\mathbf M^\top \mathbf M$ is invertible, we can choose $\Gamma$ as:

\[
\Gamma = \mathbf D (\mathbf M^\top \mathbf M)^{-1}
\]
%
where $\mathbf D > 0$. Then $-\Gamma \mathbf M^\top \mathbf M = -\mathbf D  < 0$, and the estimator will be stable.

\end{flushleft}
\end{frame}




\begin{frame}
\centerline{Lecture slides are available via Moodle.}
\bigskip
\centerline{You can help improve these slides at:}
\centerline{\url{https://github.com/SergeiSa/Linear-Control-Slides-Spring-2020}}
\bigskip
\centerline{Check Moodle for additional links, videos, textbook suggestions.}
\end{frame}

\end{document}
