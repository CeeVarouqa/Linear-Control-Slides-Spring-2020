\documentclass{article}
\usepackage[utf8]{inputenc}
\usepackage{systeme}
\title{Observer Design}
\author{Mikhail Gudkov}
\date{April 2020}

\begin{document}

\maketitle
Imagine we have the system of equations:
\[
    \systeme*{y=C x,x^{\prime}=A x + B u }
\]
Where x is, for example,vector of coordinate and velocity of some object, $x^{\prime}$ is vector of object's acceleration and velocity. u is a controller of our system, and matrices A and B are matrices of some coefficients. y is an output of the system we control.
\newline
We want to transform this system to 
$x^\prime = ( A - B K ) x$, $ A - B K \prec 0$, simply speaking, we want to make our system stable and make output approach to, for example, some constant.
\newline
Let's suppose we estimate this value x, then
$\overline{x_{i}}$ - estimated value.
\newline
Let's substitute estimated value to our equation.
\begin{equation}
    \overline{x_{i}^\prime}=A\overline{x_{i}}+Bu+L(y-\overline{y})
\end{equation}
In this equation L is a feedback, based on actual y and estimated y.
By adding feedback, we do the so called closed-loop or feedback control. Simply speaking, our system now depends on feedback and produces some actions on control element, based on this feedback. Let's consider an example.
\newline
\textbf{Example:}
We have a boiler system, which warms water to some fixed temperature. We have a thermometer, which measures temperature of water. Boiler gets feedback from thermometer and understand that it should/shouldn't continue warming water.
\newline
Previously in our course we had:
\begin{equation}
    u=-Kx
\end{equation}
We knew that u=-Kx.
But now we can not make an assumption about the form of u, so the next equation may be incorrect.
\begin{equation}
    x^{\prime}=Ax-BK\overline{x}
\end{equation}
Our goal now is to find $\overline{x} \rightarrow x$ such that equation 3 would be reliable.
"Equation number 1 gets us back on track."(c)Sergei Savin.
With this equation we get feedback from our system and now can control it.
\newline
We define estimation error. Simply speaking difference between estimated and actual values of x.
\begin{equation}
    e=\overline{x}-x 
\end{equation}
From the very first system of equations we have.
\begin{equation}
    y=Cx
\end{equation}
Let's substitute it with estimated output.
\begin{equation}
    \overline{y}=C\overline{y}
\end{equation}
Now we want to represent our equation in terms of errors, so we are going to substitute errors insted of x:
\begin{equation}
    \overline{x^{\prime}}-x^{\prime}=A\overline{x}-Ax + Bu - Bu + LCx - LC\overline{x}
\end{equation}
We subtracted Bu in the equation above in order to, again, not to make assumptions about form of u.
\begin{equation}
    e^{\prime}=Ae-LCe \Rightarrow e^{\prime}=(A-LC)e
\end{equation}
As you can see, now we have equation $x^\prime = ( A - B K ) x$ but with errors instead of x. As previously we need this system stable, so we have next formula.
\newline
$A-LC \prec 0$
\newline
To check the stability of a system, we need to find eigenvalues of A - LC.
Small reminder: if real parts of eigenvalues will be less than zero, then the system will be stable.
We had $ A - B K \prec 0$, now we have $A-LC \prec 0$. And if it was like previously, we would be able to solve it. Now we need to find another way. 
\newline
If we assume that C is identity matrix, then $A-L \prec 0$ and we get $z^{\prime}=Az+Iu$, $u=-Lz$ and we easily solve it, since $I$ is a new B and $L$ is a new K. But on practice C is almost never identity, so we need to try another approach.
\newline
$A-LC \prec 0 \Rightarrow (A-LC)^{T}\prec 0 $ Since eigenvalues will not change $\Rightarrow A^{T}-C^{T}L^{T}\prec 0$
Now, for the system of equations:
\[
\systeme{z^{\prime}=A^{T}z+c^{T}v,v=-L^{T}z}
\]
we should find controller.
$L^{T}=lqr(A^{T},C^{T},Q,R)$
Where Q and R are matrices from LQR, which was described on previous lectures.
This function will give us gaines of an obsever.
We want to minimize integral of an observation error and LQR will do it for us.
\newline
Doing described above actions we now can easily solve our problem.
\end{document}